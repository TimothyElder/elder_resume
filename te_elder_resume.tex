\documentclass{resume}

\usepackage[top=0.5in, bottom=0.5in, left=0.5in, right=0.5in]{geometry}
\usepackage{enumitem}
\usepackage{url}
\usepackage{hyperref} % Clickable links
\hypersetup{
    colorlinks=true,
    linkcolor=blue,
    filecolor=blue,
    urlcolor=blue,
    citecolor = blue
    }
  \urlstyle{same}

\def\myemail{timothyelder@uchicago.edu}

% Header
\begin{document}
\begin{center}
\thispagestyle{empty}
\large \textbf{Timothy Elder \\}
\normalsize \href{mailto:\myemail}{\myemail} $\mid$ (740) 913-1219 $\mid$ \href{http://timothyelder.com/}{timothyelder.com}   \\
\end{center}

% Education
\begin{rSection}{Education}

\textbf{University of Chicago} \hfill Chicago, Illinois\\
\textit{Phd, Sociology} \hfill Expected August, 2022\\

\textit{Focus Areas}: Sociology of Medicine, Knowledge and Science, Qualitative In-Depth Interviews, Exponential Random Graph Models, Hierarchical Linear Models, Maximum Likelihood Estimation, Ordinary Least Squares Estimation, Survey Analysis.

\textbf{Kent State University} \hfill Kent, Ohio \\
\textit{B.A. Sociology and Philosophy} \hfill May, 2016\\

\end{rSection}

% Current Projects
\begin{rSection}{Current Projects}
\noindent \textbf{An Assembly of Choice: Success, Failure and Professional Identity in Palliative Medicine}  \\
\textit{Dissertation, Department of Sociology} %\hfill Month, Year $-$ August, 2020
\begin{itemize}[noitemsep,nolistsep,leftmargin=*]
\item {Qualitative interview study which seeks to better understand how physicians in palliative medicine create a set of choices for patients experiencing life-limiting and terminal illness. }
\item { Drawing on several sociological literatures I explore how these kinds of physicians create metrics of success in conditions consider by most other kinds of physicians to be ``No win scenarios''. Further, I explore several tensions in medical practice but primarily between patient autonomy and physician expertise. \\}

\end{itemize}

\noindent \textbf{Status Hierarchies in Sociology Departments and Subfields}  \\
\textit{Ongoing Research} %\hfill Month, Year $-$ August, 2020
\begin{itemize}[noitemsep,nolistsep,leftmargin=*]
\item {Using Python, R, and Optical Character Recognition to digitize physical text, created a novel dataset of all sociologists at PhD granting departments for the last 10 years. Analyzing the data with Exponential Random Graph Models and a variety of centrality measures, examines the distribution of evaluations of prestige across research subfields and interests.}
\item {Initial findings show that research interests are stratified by attributions of ``prestige'', the perception of achievement or quality, particularly associated with race and gender with research interests of African American and Female academics receiving lower evaluations of quality, a further form of stratification and discrimination in scientific enterprises. \\}
\end{itemize}

\noindent \textbf{The Prevalence and Predictors of the Use of Advance Directives In Long Term Care
Patients} \\
\textit{Ongoing Research} %\hfill Month, Year $-$ August, 2020
\begin{itemize}[noitemsep,nolistsep,leftmargin=*]
\item {Leveraged R and STATA to analyze two waves of CDC survey data to identify predictors for when long term care patients establish an Advance Directive, a legal document which specifies the type of healthcare they want to receive in case of a medical emergency.}
\item {Utilizing logistic regression to explore several features of a patient's disease trajectory and symptom burden to better understand the circumstances in which they establish advance directives.}
\item {Research will extend findings with a second wave of data to be released in early 2021. \\}
\end{itemize}

\end{rSection}

% Teaching Assistantships
\begin{rSection}{Teaching Assistantships}
\begin{itemize}[noitemsep,nolistsep,leftmargin=*]
\item {\textit{Maverick Markets, Prof. Karin Knorr-Cetina} \hfill Spring 2020}
\item {\textit{Social Structure and Process, Prof. Marco Garrido} \hfill Winter 2019}
\item {\textit{Sociology of Human Sexuality, Prof. Edward Laumann} \hfill Spring 2019}
\end{itemize}

\end{rSection}

% Other Experiences
\begin{rSection}{Research Assistantships}
\noindent \textbf{Research Assistant} \textit{Course Development, Invitation to Sociology} \hfill  2018 $-$ 2019
\begin{itemize}[noitemsep,nolistsep,leftmargin=*]
\item { Assisted Professors Martin and Trinitapoli in the Department of Sociology at the University of Chicago to create a new course to instruct students in the fundamentals of contemporary sociological method and theory.}
\item { Reviewed recent sociological literature to synthesize the contemporary state of the discipline for the consumption by first and second year undergraduate students. Further, helped to implement the use of in-class survey tools to demonstrate sociological concepts to the students. \\}
\end{itemize}

\noindent \textbf{Research Assistant} \textit{Group Processes Lab, Kent State University} \hfill  2015 $-$ 2016
\begin{itemize}[noitemsep,nolistsep,leftmargin=*]
\item {Research assistant in the Group Processes Lab in Kent State University’s Department of Sociology. Focused on the interdisciplinary Electrophysiological Neuroscience Laboratory, a laboratory for the collection of data related to brain function, where I collaborated with other team members to collect neurophysiological data for various research projects at Kent State University.}
\item {Completed IRB training in Human Subjects Research and supervised the application EEG caps and saline solutions to research subject's scalp. Monitored subject's brainwaves and EEG equipment as research subjects completed research tasks.\\}
\end{itemize}

\noindent \textbf{Research Assistant} \textit{Empirical Investigations Into Attitudes Toward Privacy} \hfill 2014$-$ 2016
\begin{itemize}[noitemsep,nolistsep,leftmargin=*]
\item {Collaborated with Prof. Richard Serpe, Chair of the Department of Sociology on a survey research project which sought to investigate the attitudes and behaviors of Americans towards privacy and private information in their everyday lives.}
\item {Developed a new survey tool, with a wide range of psychometrics to measure dispositions to sharing or protecting information, as well as political attitudes regarding the American governments use and disclosure of private information. Deployed and collected the new survey, analyzed the data with STATA and presented the findings at the 2015 Meeting of the American Sociological Association. \\}
\end{itemize}

\end{rSection}


% Other Skills
\noindent \textbf{\underline{Technical Skills}} \\
\noindent R for statistical analyses and Social Network Analysis. Python for Natural Language Processing, Web Scraping, Social Network Analysis, and General Programming. \\


\end{document}
